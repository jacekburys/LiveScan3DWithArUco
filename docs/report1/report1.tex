\documentclass[a4paper,12pt]{article}
\usepackage [utf8]{inputenc}
\usepackage{listings}
\usepackage{color}

\definecolor{dkgreen}{rgb}{0,0.6,0}
\definecolor{gray}{rgb}{0.5,0.5,0.5}
\definecolor{mauve}{rgb}{0.12,0.37,0.12}
\definecolor{darkblue}{rgb}{0,0,0.5}

\lstset{frame=tb,
  language=C,
  aboveskip=3mm,
  belowskip=3mm,
  showstringspaces=false,
  columns=flexible,
  basicstyle={\small\ttfamily},
  numbers=none,
  numberstyle=\tiny\color{black},
  keywordstyle=\color{darkblue},
  commentstyle=\color{gray},
  stringstyle=\color{mauve},
  breaklines=true,
  breakatwhitespace=true,
  tabsize=3
}

\newcommand*\justify{%
  \fontdimen2\font=0.4em% interword space
  \fontdimen3\font=0.2em% interword stretch
  \fontdimen4\font=0.1em% interword shrink
  \fontdimen7\font=0.1em% extra space
  \hyphenchar\font=`\-% allowing hyphenation
}

\addtolength{\hoffset}{-1,5cm}
\addtolength{\textwidth}{3cm}

\title{LiveScan 3D With ArUco Markers}
\author{Rui Liu, Adam Hosier, Jacek Burys,\\ Ayman Moussa, Kabeer Vohra}
\date{\today}

\begin{document}

\maketitle

Introduction

\section*{Methods}
For this project we have decided to follow the Extreme Programming (XP) agile method of development. The reasons for this are:

\begin{itemize}
\item Extreme programming is split up into iterations which works well with the checkpoints we need to deliver throughout the project. We will do smaller iterations in between the project checkpoint cycles to ensure we are on track to meet the next larger checkpoint.
\item Creating an open work space for extreme programming will be easier to achieve while doing pair programming as we will need to work with the Kinect camera hardware so will likely be meeting to work together in labs for large portions of the project.
\item Pair programming will be useful as well because we need to be working with the hardware which will be in labs so we can maximise the time we have with the hardware and work together rather than remotely.
\item Ensuring the customer is always available is also convenient for this project as our customer is our supervisor. We will do requirement capture via user stories from our supervisor in order to help guide the development and ensure all conditions are met.
\item Daily stand-up meetings will be possible in since we are in university in order to keep track of progress within the project and ensure we are on track for delivering the requirements.
\end{itemize}

\section*{Planning}
Hello

\end{document}
